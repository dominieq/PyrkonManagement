\documentclass[11pt]{article}

\usepackage[T1]{fontenc}
\usepackage[polish]{babel}
\usepackage[utf8]{inputenc}
\usepackage{lmodern}
\selectlanguage{polish}

\title{Przetwarzanie Rozproszone -- Obsługa Pyrkonu}
\author{Wojciech Kulig (115881), Dominik Szmyt (132326)}
\date{11.09.2019}

\usepackage{natbib}
\usepackage{graphicx}
\usepackage{subcaption}
\usepackage{float}

\begin{document}

\maketitle

\section{Opis problemu}

\subsection{Krótki opis}
~~~~Proces realizujący program obsługi Pyrkonu może znajdować się w czterech stanach (poniżej tylko krótki opis):
\begin{enumerate}
\item Przed Pyrkonem (BEFORE\_PYRKON) - Proces ,,czeka w kolejce, żeby uczestniczyć w aktywnościach na Pyrkonie''.
\item Na Pyrkonie (ON\_PYRKON) - Proces ,,jest już na Pyrkonie, może uczestniczyć w warsztatach''.
\item Po Pyrkonie (AFTER\_PYRKON) - Proces ,,wyszedł z Pyrkonu i czeka na inne procesy żeby rozpocząć zabawę na nowo''.
\end{enumerate}
~~~~Procesy wymieniją między sobą następujące rodzaje komunikatów:
\begin{enumerate}
\item WANT\_TO\_ENTER - Proces informuje, że chce dostać się na Pyrkon (komunikat z argumentem 0) lub na któryś z ,,n'' warsztatów (liczby od 1 do n).
\item ALRIGHT\_TO\_ENTER - Proces pozwala innemu procesowi na wejście na Pyrkon lub któryś z warsztatów.
\item EXIT - Proces informuje inne procesy, że wyszedł już z Pyrkonu.
\end{enumerate}

\subsection{Długi Opis} 
Proces chcąc dostać się na Pyrkon informuje o tym wszystkie pozostałe procesy, stając w ten sposób w kolejce. Otrzymujący zapytanie o zgodę na wejście na Pyrkon proces może wyrazić zgodę (jeśli sam się właśnie nie ubiega i nie uczestniczy lub znajduje się w kolejce za pytającym procesem) lub póki co nie odpowiadać. Kiedy proces uzyska odpowiednią liczbę zgód, gwarantującyh istnienie dla niego miejsca na Pyrkonie, wchodzi (warunek: liczba\_otrzymanych\_zgód >= liczba\_procesów - maksymalna\_liczba\_procesów\_na\_Pyrkonie).\\
Kiedy proces wychodzi z Pyrkonu zwalnia miejsce, odpowiadając na wcześniejsze zapytania, na które nie udzielił jeszcze odpowiedzi.\\
Sytacja jest analogiczna dla warsztatów.\\
Wychodząc z Pyrkonu proces informuje o tym wszystkie pozostałe. Kiedy proces będzie po Pyrkonie i otrzyma łącznie n-1 (gdzie n to liczba procesów) informacji o wyjściu procesów z Pyrkonu, rozpoczał się nowy.\\

\section{Złożoność czasowa i komunikacyjna}
\textbf{Złożoność czasowa} jest funkcją kosztu wykonania algorytmu rozproszonego, wyrażoną przez liczbę kroków algorytmu do jego zakończenia przy założeniu, że:
\begin{itemize}
\item czas wykonywania każdego kroku (operacji) jest stały
\item kroki wykonywane są synchronicznie
\item czas transmisji wiadomości jest stały
\end{itemize}
Przyjmuje się też na ogół, że:
\begin{itemize}
\item czas przetwarzania lokalnego (wykonania każdego kroku) jest pomijalny (zerowy)
\item czas transmisji jest jednostkowy
\end{itemize}
\medskip 
\textbf{Złożoność komunikacyjna} pakietowa jest funkcją kosztu wykonania algorytmu wyrażaną przez liczbę pakietów (wiadomości) przesyłanych w trakcie wykonywania algorytmu do jego zakończenia.

\bigskip 

Algorytm można podzielić na trzy części odpowiadające kolejnym zadaniom realizowanym przez zaproponowany algorytm:
\begin{enumerate}
\item wejście na Pyrkon - od podjęcia decyzji o wejściu na Pyrkon do faktycznego wejścia:
\begin{itemize}
\item Proces wysyła do wszystkich informację, że chce się dostać na Pyrkon. (czas: 1; komunikatów: n-1)
\item Otrzymujący informację proces może wyrazić zgodę lub nie odpowiadać jeszcze, lecz w końcu to zrobi. (czas: 1; komunikatów: n-1)
\item Po otrzymaniu odpowiedniej liczby zgód (liczba zgód gwarantująca, że jego wejście nie przekroczy maksymalnej ilości uczestników na Pyrkonie) proces wchodzi.\\
Pozostałe odpowiedzi otrzymuje, nie wpływają one na przetwarzanie.
\end{itemize}
% TODO ??
Szacowana złożoność to:
\begin{itemize}
\item czasowa: 2
\item komunikacyjna: 2*(n-1)
\end{itemize}
\item wejście na Warsztat - analogiczne jak dla wejścia na Pyrkon
\item nowy Pyrkon
\begin{itemize}
\item Proces wychodzący z Pyrkonu informuje o tym wszystkie pozostałe procesu. Nowy Pyrkon rozpocznie się w momencie, kiedy liczba zebranych informacji o wyjściu z Pyrkonu będzie równa liczbie procesów.
\end{itemize}
Szacowana złożoność to:
\begin{itemize}
\item czasowa: 1
\item komunikacyjna: n-1
\end{itemize}
\end{enumerate}

\end{document}
