\documentclass[11pt]{article}

\usepackage[T1]{fontenc}
\usepackage[polish]{babel}
\usepackage[utf8]{inputenc}
\usepackage{lmodern}
\selectlanguage{polish}

\title{Przetwarzanie Rozproszone -- Obsługa Pyrkonu}
\author{Wojciech Kulig (115881), Dominik Szmyt (132326)}
\date{11.09.2019}

\usepackage{natbib}
\usepackage{graphicx}
\usepackage{subcaption}
\usepackage{float}

\begin{document}

\maketitle

\section{Opis problemu}

Proces realizujący program obsługi Pyrkonu może znajdować się w czterech stanach (poniżej tylko krótki opis):
\begin{enumerate}
\item Przed Pyrkonem - Proces "czeka w kolejce, żeby uczestniczyć w aktywnościach na Pyrkonie.
\item Na Pyrkonie - Proces "jest już na Pyrkonie, ale nie uczestniczy w warsztatach".
\item Na warsztacie - Proces "jest na jednym ze swoich warsztatów przez jakiś czas".
\item Po Pyrkonie - Proces "wyszedł z Pyrkonu i czeka na inne procesy żeby rozpocząć zabawę na nowo". 
\end{enumerate}


\end{document}