\documentclass[11pt]{article}

\usepackage[T1]{fontenc}
\usepackage[polish]{babel}
\usepackage[utf8]{inputenc}
\usepackage{lmodern}
\selectlanguage{polish}

\title{Przetwarzanie Rozproszone -- Obsługa Pyrkonu}
\author{Wojciech Kulig (115881), Dominik Szmyt (132326)}
\date{11.09.2019}

\usepackage{natbib}
\usepackage{graphicx}
\usepackage{subcaption}
\usepackage{float}

\begin{document}

\maketitle

\section{Opis problemu}

\subsection{Krótki opis}
~~~~Proces realizujący program obsługi Pyrkonu może znajdować się w czterech stanach (poniżej tylko krótki opis):
\begin{enumerate}
\item Przed Pyrkonem - Proces ,,czeka w kolejce, żeby uczestniczyć w aktywnościach na Pyrkonie''.
\item Na Pyrkonie - Proces ,,jest już na Pyrkonie, ale nie uczestniczy w warsztatach''.
\item Na warsztacie - Proces ,,jest na jednym ze swoich warsztatów przez jakiś czas''.
\item Po Pyrkonie - Proces ,,wyszedł z Pyrkonu i czeka na inne procesy żeby rozpocząć zabawę na nowo''.
\end{enumerate}

\subsection{Długi Opis} 

\section{Złożoność czasowa i komunikacyjna}
\textbf{Złożoność czasowa} jest funkcją kosztu wykonania algorytmu rozproszonego, wyrażoną przez liczbę kroków algorytmu do jego zakończenia przy założeniu, że:
\begin{itemize}
\item czas wykonywania każdego kroku (operacji) jest stały
\item kroki wykonywane są synchronicznie
\item czas transmisji wiadomości jest stały
\end{itemize}
Przyjmuje się też na ogół, że:
\begin{itemize}
\item czas przetwarzania lokalnego (wykonania każdego kroku) jest pomijalny (zerowy)
\item czas transmisji jest jednostkowy
\end{itemize}
\medskip 
\textbf{Złożoność komunikacyjna} pakietowa jest funkcją kosztu wykonania algorytmu wyrażaną przez liczbę pakietów (wiadomości) przesyłanych w trakcie wykonywania algorytmu do jego zakończenia.

\bigskip 

Algorytm można podzielić na trzy części odpowiadające kolejnym zadaniom realizowanym przez zaproponowany algorytm:
\begin{enumerate}
\item wejście na Pyrkon - od podjęcia decyzji o wejściu na Pyrkon do faktycznego wejścia:
\begin{enumerate}
\item Proces wysyła do wszystkich informację, że chce się dostać na Pyrkon. (czas: 1; komunikatów: n-1)
\item Otrzymujący informację proces może wyrazić zgodę lub nie odpowiadać jeszcze, lecz w końcu to zrobi. (czas: 1; komunikatów: n-1)
\item Po otrzymaniu odpowiedniej liczby zgód (liczba zgód gwarantująca, że jego wejście nie przekroczy maksymalnej ilości uczestników na Pyrkonie) proces wchodzi.\\
Pozostałe odpowiedzi otrzymuje, nie wpływają one na przetwarzanie.\\
Pesymistycznie, proces próbujący się dostać na Pyrkon od samego początku (rozpocznie razem z innymi) wejdzie jako ostatni.
\end{enumerate}
% TODO ??
Szacowana złożoność to:
\begin{itemize}
\item czasowa: 2
\item komunikacyjna: n-1
\end{itemize}
\item wejście na Warsztat - analogiczne jak dla wejścia na Pyrkon
\item nowy Pyrkon
\begin{enumerate}
\item Proces wychodzący z Pyrkonu informuje o tym wszystkie pozostałe procesu. Nowy Pyrkon rozpocznie się w momencie, kiedy liczba zebranych informacji o wyjściu z Pyrkonu będzie równa liczbie procesów.
\end{enumerate}
Szacowana złożoność to:
\begin{itemize}
\item czasowa: 1
\item komunikacyjna: n-1
\end{itemize}
\end{enumerate}

\end{document}